\documentclass[12pt]{article}
\usepackage[left=2cm,right=2cm,top=2cm,bottom=2cm,bindingoffset=0cm]{geometry}
\usepackage[utf8x]{inputenc}
\usepackage[english,russian]{babel}
\usepackage{cmap}
\usepackage{amssymb}
\usepackage{amsmath}
\usepackage{url}
\usepackage{pifont}
\usepackage{tikz}
\usepackage{verbatim}

\usetikzlibrary{shapes,arrows}
\usetikzlibrary{positioning,automata}
\tikzset{every state/.style={minimum size=0.2cm},
initial text={}
}


\newenvironment{myauto}[1][3]
{
  \begin{center}
    \begin{tikzpicture}[> = stealth,node distance=#1cm, on grid, very thick]
}
{
    \end{tikzpicture}
  \end{center}
}


\begin{document}
\begin{center} {\LARGE Формальные языки} \end{center}

\begin{center} \Large домашнее задание до 23:59 05.03 \end{center}
\bigskip

\begin{enumerate}
  \item Доказать или опровергнуть утверждение: произведение двух минимальных автоматов всегда дает минимальный автомат (рассмотреть случаи для пересечения, объединения и разности языков).
  
  {\bf Решение:} Возьмем два автомата, один из которых принимает бинарные слова, 
  оканчивающиеся на 0, а другой --- оканчивающиеся на 1. Их минимальные автоматы,
  очевидно, выглядят так:
  \begin{enumerate}
  \item
  
  \begin{tikzpicture}[->,>=stealth',shorten >=1pt,auto,node distance=3.5cm,
        scale = 1,transform shape]

  \node[state,initial] (A) {$A$};
  \node[state,accepting] (B) [right=of A] {$B$};

  \path (A) edge  [bend right=15]             node [below] {$0$} (B)
        (A) edge    [loop above]          node  {$1$} (A)
        (B) edge     [loop above]         node  {$0$} (B)
        (B) edge  [bend right=15]            node [above] {$1$} (A);

\end{tikzpicture}
  
  \item
  
  \begin{tikzpicture}[->,>=stealth',shorten >=1pt,auto,node distance=3.5cm,
        scale = 1,transform shape]

  \node[state,initial] (a) {$a$};
  \node[state,accepting] (b) [right=of a] {$b$};

  \path (a) edge   [bend right=15]            node [below]{$1$} (b)
        (a) edge    [loop above]          node  {$0$} (a)
        (b) edge     [loop above]         node  {$1$} (b)
        (b) edge   [bend right=15]           node [above] {$0$} (a);

\end{tikzpicture}

  \end{enumerate}
  
  Возьмем их произведение:
  
  \begin{enumerate}
  \item Пересечение:
  
  \begin{tikzpicture}[->,>=stealth',shorten >=1pt,auto,node distance=3.5cm,
        scale = 1,transform shape]

  \node[state,initial] (Aa) {$Aa$};
  \node[state] (Ab) [right of = Aa] {$Ab$};
  \node[state] (Ba) [below of= Aa] {$Ba$};
  \node[state,accepting] (Bb) [below of= Ab] {$Bb$};

  \path (Aa) edge              node {$1$} (Ab)
        (Ab) edge    [loop above]          node {$1$} (Ab)
        (Aa) edge              node {$0$} (Ba)
        (Ab) edge  [bend left = 10]            node {$0$} (Ba)
        (Ba) edge    [loop below]        node {$0$} (Ba)
        (Ba) edge     [bend left = 10]           node {$1$} (Ab)
        (Bb) edge              node {$1$} (Ab)
        (Bb) edge              node {$0$} (Ba);

\end{tikzpicture}

Но минимальный автомат должен быть таким (то есть у нас пересечение пустое и нет
терминальной вершины):
    \begin{tikzpicture}[->,>=stealth',shorten >=1pt,auto,node distance=3.5cm,
        scale = 1,transform shape]

  \node[state,initial] (X) {$X$};

  \path (X) edge    [loop above]          node {$0,1$} (X);

    \end{tikzpicture}
    
    \item Объединение: 
    
    \begin{tikzpicture}[->,>=stealth',shorten >=1pt,auto,node distance=3.5cm,
        scale = 1,transform shape]

  \node[state,initial] (Aa) {$Aa$};
  \node[state,accepting] (Ab) [right of = Aa] {$Ab$};
  \node[state,accepting] (Ba) [below of= Aa] {$Ba$};
  \node[state,accepting] (Bb) [below of= Ab] {$Bb$};

  \path (Aa) edge              node {$1$} (Ab)
        (Ab) edge    [loop above]          node {$1$} (Ab)
        (Aa) edge              node {$0$} (Ba)
        (Ab) edge  [bend left = 10]            node {$0$} (Ba)
        (Ba) edge    [loop below]        node {$0$} (Ba)
        (Ba) edge     [bend left = 10]           node {$1$} (Ab)
        (Bb) edge              node {$1$} (Ab)
        (Bb) edge              node {$0$} (Ba);

\end{tikzpicture}
     
     Но минимальный автомат должен быть таким (то есть у нас объединение это все 
     непустые бинарные слова):
     
     \begin{tikzpicture}[->,>=stealth',shorten >=1pt,auto,node distance=3.5cm,
        scale = 1,transform shape]

  \node[state,initial] (X) {$X$};
  \node[state,accepting] (Y) [right of = X] {$Y$};

  \path (X) edge             node {$0,1$} (Y)
           (Y) edge    [loop above]          node {$0,1$} (Y);

    \end{tikzpicture}

    \item Разность первого и второго языка --- это пустое множество, поэтому минимальный
    автомат как в пересечении, а произведение без терминальных вершин, то есть такое:
    
    \begin{tikzpicture}[->,>=stealth',shorten >=1pt,auto,node distance=3.5cm,
        scale = 1,transform shape]

  \node[state,initial] (Aa) {$Aa$};
  \node[state] (Ab) [right of = Aa] {$Ab$};
  \node[state] (Ba) [below of= Aa] {$Ba$};
  \node[state] (Bb) [below of= Ab] {$Bb$};

  \path (Aa) edge              node {$1$} (Ab)
        (Ab) edge    [loop above]          node {$1$} (Ab)
        (Aa) edge              node {$0$} (Ba)
        (Ab) edge  [bend left = 10]            node {$0$} (Ba)
        (Ba) edge    [loop below]        node {$0$} (Ba)
        (Ba) edge     [bend left = 10]           node {$1$} (Ab)
        (Bb) edge              node {$1$} (Ab)
        (Bb) edge              node {$0$} (Ba);

\end{tikzpicture}

    \end{enumerate}


  \item Для регулярного выражения:
   \[ (a \mid b)^+ (aa \mid bb \mid abab \mid baba)^* (a \mid b)^+\]
  Построить эквивалентные:
  \begin{enumerate}
    \item Недетерминированный конечный автомат
    
    \begin{tikzpicture}[->,>=stealth',shorten >=1pt,auto,node distance=3.5cm,
        scale = 1,transform shape]

  \node[state,initial] (A) {$A$};
  \node[state] (B) [right of=A] {$B$};
  \node[state] (C) [right of=B] {$C$};
  \node[state] (D) [above left of=C] {$D$};
  \node[state] (E) [above right of=D] {$E$};
  \node[state] (F) [above right of=C] {$F$};
  \node[state] (G) [below left of=C] {$G$};
  \node[state] (H) [below right of=G] {$H$};
  \node[state] (I) [below right of=C] {$I$};
  \node[state] (J) [right of=C] {$J$};
  \node[state,accepting] (K) [right of=J] {$K$};

  \path (A) edge              node {$a, b$} (B)
        (B) edge   [loop above]           node {$a, b$} (B)
        (B) edge              node {$\varepsilon$} (C)
        (C) edge  [bend left=10]            node {$a$} (D)
        (D) edge  [bend left=10]            node {$a$} (C)
        (D) edge              node {$b$} (E)
        (E) edge              node {$a$} (F)
        (F) edge              node {$b$} (C)
        (C) edge  [bend left=10]             node {$b$} (G)
        (G) edge  [bend left=10]             node {$b$} (C)
        (G) edge              node {$a$} (H)
        (H) edge              node {$b$} (I)
        (I) edge              node {$a$} (C)
        (C) edge              node {$\varepsilon$} (J)
        (J) edge  [loop above]             node {$a, b$} (J)
        (J) edge              node {$a, b$} (K);

\end{tikzpicture}

    \item Недетерминированный конечный автомат без $\varepsilon$-переходов
    
    \begin{tikzpicture}[->,>=stealth',shorten >=1pt,auto,node distance=3.5cm,
        scale = 1,transform shape]

  \node[state,initial] (A) {$A$};
  \node[state] (B) [right of=A] {$B$};
  \node[state] (C) [right of=B] {$C$};
  \node[state] (D) [above left of=C] {$D$};
  \node[state] (E) [above right of=D] {$E$};
  \node[state] (F) [above right of=C] {$F$};
  \node[state] (G) [below left of=C] {$G$};
  \node[state] (H) [below right of=G] {$H$};
  \node[state] (I) [below right of=C] {$I$};
  \node[state,accepting] (K) [right of=J] {$K$};

  \path (A) edge              node {$a, b$} (B)
        (B) edge   [loop below]           node {$a, b$} (B)
        (B) edge              node {$a$} (D)
        (B) edge              node {$b$} (G)
        (B) edge   [bend right=15]           node {$a,b$} (K)
        (C) edge  [bend left=10]            node {$a$} (D)
        (D) edge  [bend left=10]            node {$a$} (C)
        (D) edge              node {$b$} (E)
        (E) edge              node {$a$} (F)
        (F) edge              node {$b$} (C)
        (C) edge  [bend left=10]             node {$b$} (G)
        (G) edge  [bend left=10]             node {$b$} (C)
        (G) edge              node {$a$} (H)
        (H) edge              node {$b$} (I)
        (I) edge              node {$a$} (C)
        (K) edge [loop above] node {$a,b$} (K)
        (C) edge              node {$a,b$} (K);

\end{tikzpicture}
    
    \item Минимальный полный детерминированный конечный автомат
    
    Можно присмотреться на определение регулярного языка или на предыдущий пункт и понять, что это просто все
    такие слова, в которых хотя бы 2 символа (то есть можно просто убрать вершину  C
    и два ромбика, который сверху и снизу) и получается:
    
    \begin{tikzpicture}[->,>=stealth',shorten >=1pt,auto,node distance=3.5cm,
        scale = 1,transform shape]

  \node[state,initial] (A) {$A$};
  \node[state] (B) [right of=A] {$B$};
  \node[state,accepting] (K) [right of=B] {$K$};

  \path (A) edge              node {$a, b$} (B)
        (B) edge         node {$a,b$} (K)
        (K) edge [loop above] node {$a,b$} (K);

\end{tikzpicture}
    
  \end{enumerate}
  \item Построить регулярное выражение, распознающее тот же язык, что и автомат:
  \begin{myauto}
    \node[state]           (q_2)                {$q_2$};
    \node[state,initial]   (q_0) [left=of  q_2] {$q_0$};
    \node[state]           (q_1) [above=of q_2] {$q_1$};
    \node[state]           (q_3) [below=of q_2] {$q_3$};
    \node[state,accepting] (q_4) [right=of q_2] {$q_4$};

    \path[->] (q_0) edge [loop above] node [above] {$a, b, c$} ()
                    edge              node [above] {$a$}       (q_1)
                    edge              node [above] {$b$}       (q_2)
                    edge              node [above] {$c$}       (q_3)
              (q_1) edge [loop above] node [above] {$b, c$}    ()
                    edge              node [above] {$a$}       (q_4)
              (q_2) edge [loop above] node [above] {$a, c$}    ()
                    edge              node [above] {$b$}       (q_4)
              (q_3) edge [loop above] node [above] {$a, b$}    ()
                    edge              node [above] {$c$}       (q_4)
    ;
  \end{myauto}
  
  {\bf Решение: } $(a|b|c)^*((a(b|c)^*a)\ |\ (b(a|c)^*b)\ |\ (c(a|b)^*c))$
  
  \item Определить, является ли автоматным язык $\{ \omega \omega^r \mid \omega \in \{ 0, 1 \}^* \}$. Если является --- построить автомат, иначе --- доказать.
  
  {\bf Решение: }
  Обозначим этот язык за $L$ и предположим, что он автоматный. Возьмем $n$ из леммы о накачке и $w = 01^{n}001^{n}0$. Длина этого слова не меньше $n$ и оно лежит в $L$, поэтому из леммы верно, что $ \exists x,y,z \in \Sigma^* :\ xyz=w, |y| \ne \varepsilon,\ |xy| \le n$, т.ч.
$\forall k\  xy^k z \in L $. Возможны два случая:
\begin{enumerate}
   \item $x = \varepsilon$ (пустое слово), тогда в $y$ будет ровно 1 ноль, так как есть ограничение на длину, а в $z$ их тогда будет 3. Возьмем $k=2$ и получим, что в $xy^kz$ будет нечетное количество нулей (5), поэтому  это слово точно не будет в $L$.
   \item $x \ne \varepsilon$. Возьмем $k = 3$. Заметим, что второй ноль в $w$ по построению должен быть в $z$, поэтому он будет после $xy^k$ (в $xy^k$ только один 0 и он в $x$, так как оно не пусто). Чтобы $xy^kz$ было в $L$ как минимум нужно, чтобы в левой и правой половине слова было по 2 нуля, но такое невозможно, потому что второй ноль стоял на позиции $n+2$ в $w$, а в таком слове на $n+2+2|y|$, а середина слова была между $n+2$ и $n+3$, а стала между $n+ 2 + |y|$ и $n+3+|y|$, то есть левее второго нуля. Поэтому такой случай тоже приводит к противоречию.
  \end{enumerate}
  Язык не автоматный.

  
  \item Определить, является ли автоматным язык $\{ u a a v \mid u, v \in \{ a, b \}^* , |u|_b \geq |v|_a \}$. Если является --- построить автомат, иначе --- доказать.
  
  {\bf Решение: } Не является. Докажем это также через лемму о накачке: Пусть наш язык
  $L$ автоматный. Берем $n$ из леммы и $w = b^naa(ba)^n$, оно лежит в $L$. Но если мы 
  возьмем $k = 0$, то, поскольку $y$ по условию непуст, до $aa$ станет $< n$ нулей (в 
  строке $xy^kz$ возможно только одно вхождение $aa$ для любых $x, y, z$), а после $aa$ будет $n$ символов $a$. Получили противоречие с тем, что $\forall k\ xy^kz\in L$.
  
\end{enumerate}

\newpage

\begin{center}
  \Large{Пример применения алгоритма минимизации}
\end{center}

\bigskip

Минимизируем данный автомат:

\begin{center}
  \begin{tikzpicture}[> = stealth,node distance=3cm, on grid]
    \node[state]           (q_2)                      {C};
    \node[state,initial]   (q_0) [above left=of q_2]  {A};
    \node[state]           (q_1) [below left=of q_2]  {B};
    \node[state]           (q_3) [right=of q_2]       {D};
    \node[state]           (q_4) [above right=of q_3] {E};
    \node[state,accepting] (q_5) [below right=of q_3] {F};
    \node[state,accepting] (q_6) [above right=of q_5] {G};

    \path[->] (q_0) edge [bend left=15]  node [right] {$1$} (q_1)
                    edge                 node [above] {$0$} (q_2)
              (q_1) edge [bend left=15]  node [left]  {$1$} (q_0)
                    edge                 node [below] {$0$} (q_2)
              (q_2) edge [bend right=15] node [below] {$1$} (q_3)
                    edge [bend left=15]  node [above] {$0$} (q_3)
              (q_3) edge                 node [below] {$1$} (q_5)
                    edge                 node [above] {$0$} (q_4)
              (q_4) edge                 node [above] {$1$} (q_6)
                    edge                 node [right] {$0$} (q_5)
              (q_5) edge [loop below]    node         {$1$} ()
                    edge [loop left]     node         {$0$} ()
              (q_6) edge                 node [below] {$1$} (q_5)
                    edge [loop right]    node         {$0$} ();
  \end{tikzpicture}
\end{center}

Автомат полный, в нем нет недостижимых вершин --- продолжаем.

Строим обратное $\delta$ отображение.

\begin{tabular}{c|c|c}
$\delta^{-1}$ & 0 & 1 \\ \hline
A & --- & B \\
B & --- & A \\
C & A B & --- \\
D & C & C \\
E & D & --- \\
F & E F & D F G \\
G & G & E
\end{tabular}

Отмечаем в таблице и добавляем в очередь пары состояний, различаемых словом $\varepsilon$: все пары, один элемент которых --- терминальное состояние, а второй --- не терминальное состояние. Для данного автомата это пары

$(A, F), (B, F), (C, F), (D, F), (E,F), (A, G), (B, G), (C, G), (D, G), (E, G)$

Дальше итерируем процесс определения неэквивалентных состояний, пока очередь не оказывается пуста.

$(A, F)$ не дает нам новых неэквивалентных пар. Для $(B, F)$ находится 2 пары: $(A, D), (A, G)$. Первая пара не отмечена в таблице --- отмечаем и добавляем в очередь. Вторая пара уже отмечена в таблице, значит, ничего делать не надо. Переходим к следующей паре из очереди. Итерируем дальше, пока очередь не опустошится.

Результирующая таблица (заполнен только треугольник, потому что остальное симметрично) и порядок добавления пар в очередь.

\begin{tabular}{c|cc|cc|cc|c}
& A & B & C & D & E & F & G \\ \hline
A &&&&&&& \\
B &&&&&&& \\ \hline
C & \checkmark & \checkmark &&&&& \\
D & \checkmark & \checkmark & \checkmark &&&& \\ \hline
E & \checkmark & \checkmark & \checkmark & \checkmark &&& \\
F & \checkmark & \checkmark & \checkmark & \checkmark & \checkmark && \\ \hline
G & \checkmark & \checkmark & \checkmark & \checkmark & \checkmark && \\
\end{tabular}

Очередь:

$
(A, F), (B, F), (C, F), (D, F), (E,F), (A, G), (B, G), (C, G), (D, G), (E, G),
$

$
(B, D), (A, D), (A, E), (B, E), (C, E), (C, D), (D, E), (A,C), (B, C))
$

В таблице выделились классы эквивалентных вершин: $\{A, B\}, \{C\}, \{D\}, \{E\}, \{F,G\}$. Остается только нарисовать результирующий автомат с вершинами-классами. Переходы добавляются тогда, когда из какого-нибудь состояния первого класса есть переход в какое-нибудь состояние второго класса. Минимизированный автомат:

\begin{center}
  \begin{tikzpicture}[> = stealth,node distance=3cm, on grid]
    \node[state,initial]   (q_01)                     {AB};
    \node[state]           (q_2)  [right=of q_01]      {C};
    \node[state]           (q_3)  [right=of q_2]       {D};
    \node[state]           (q_4)  [above right=of q_3] {E};
    \node[state,accepting] (q_56) [below right=of q_3] {FG};

    \path[->] (q_01) edge [loop above]    node [above] {$1$} ()
                     edge                 node [above] {$0$} (q_2)
              (q_2)  edge [bend right=15] node [below] {$1$} (q_3)
                     edge [bend left=15]  node [above] {$0$} (q_3)
              (q_3)  edge                 node [below] {$1$} (q_56)
                     edge                 node [above] {$0$} (q_4)
              (q_4)  edge [bend right=15] node [left]  {$1$} (q_56)
                     edge [bend left=15]  node [right] {$0$} (q_56)
              (q_56) edge [loop below]    node         {$1$} ()
                     edge [loop left]     node         {$0$} ();
  \end{tikzpicture}
\end{center}

\end{document}
