\documentclass[12pt]{article}
\usepackage[left=2cm,right=2cm,top=2cm,bottom=2cm,bindingoffset=0cm]{geometry}
\usepackage[utf8x]{inputenc}
\usepackage[english,russian]{babel}
\usepackage{cmap}
\usepackage{amssymb}
\usepackage{amsmath}
\usepackage{url}
\usepackage{pifont}
\usepackage{tikz}
\usepackage{verbatim}

\usetikzlibrary{shapes,arrows}
\usetikzlibrary{positioning,automata}
\tikzset{every state/.style={minimum size=0.2cm},
initial text={}
}


\newenvironment{myauto}[1][3]
{
  \begin{center}
    \begin{tikzpicture}[> = stealth,node distance=#1cm, on grid, very thick]
}
{
    \end{tikzpicture}
  \end{center}
}


\begin{document}
\begin{center} {\LARGE Формальные языки} \end{center}

\begin{center} \Large домашнее задание до 23:59 26.03 \end{center}
\bigskip

\begin{enumerate}
  \item[2.]
  
	 Hачальная грамматика:
		\begin{itemize}
			\item $S \to R S \ | \ R$
			\item $R \to a S b \ | \ c R d \ | \ a b \ | \ c d \ | \ \varepsilon$
		\end{itemize}
	 
	 Избавимся от правил, в которых есть терминалы и нетерминалы и избавимся от 
	 длинных правил :
 		\begin{itemize}
 			\item $S \to R S \ | \ R$
 			\item $R \to A S_1 \ | \ C S_2 \ | \ A B \ | \ C D \ | \ \epsilon$
			\item $A \to a, B \to b, C \to c, D \to d$
  			\item $S_1 \to S B$
			\item $S_2 \to R D$
 		\end{itemize}
		
 	Удалим $\varepsilon$-правила. Нетерминалы S и R являются $\epsilon$-порождающими, поэтому для них добавим нужные правила:
		\begin{itemize}
			\item $S \to R S \ | \ R \ | \ \varepsilon$
			\item $R \to A S_1 \ | \ C S_2 \ | \ A B \ | \ C D$
			\item $A \to a, B \to b, C \to c, D \to d$
			\item $S_1 \to S B \ | \ B$
			\item $S_2 \to R D \ | \ D$
		\end{itemize}
		
	Добавляем новое стартовое:
		\begin{itemize}
			\item $S' \to S \ | \ \varepsilon$
			\item $S \to R S \ | \ R$
			\item $R \to A S_1 \ | \ C S_2 \ | \ A B \ | \ C D$
			\item $A \to a, B \to b, C \to c, D \to d$
			\item $S_1 \to S B \ | \ B$
			\item $S_2 \to R D \ | \ D$
		\end{itemize}
		
	\item [4.] Убираем унарные правила:
		\begin{itemize}
			\item $S' \to R S \ | \ A S_1 \ | \ C S_2 \ | \ A B \ | \ C D \ | \ \varepsilon$
			\item $S \to R S \ | \ A S_1 \ | \ C S_2 \ | \ A B \ | \ C D$
			\item $R \to A S_1 \ | \ C S_2 \ | \ A B \ | \ C D$
			\item $A \to a; \ B \to b; \ C \to c; \ D \to d$
			\item $S_1 \to S B \ | \ B$
			\item $S_2 \to R D \ | \ D$
		\end{itemize}
	Всё.
 
 \item[3.] Язык $\{ a^m b^n\,\,|\,\,m + n > 0, m + n\,\,\vdots\,\, 2 \}$ явяется контекстно-свободным, грамматика:

	$S \to a a S\ |\ S b b\ |\ a S b\ |\ a b\ |\ a a\ |\ b b$

	Понятно, что эти правила описывают язык из строк, в которых после $a$ идут $b$ и их всего четное число (но не 0), потому что после применения любого правила их число увеличивается на 2

	Как получать строки из данного языка? Пусть $a^mb^n$ ($m+n > 0$ и $\vdots \,\, 2$):

	\begin{enumerate}
		\item Если $n=0$, то $m$ кратно $2$, поэтому можно просто применить 
		$S \to a a S$ $\frac{m}{2} - 1$ раз и правило $S \to a a$
		\item Если $m=0$, то аналогично
		\item Если $n \neq 0, m \neq 0$, и они четные, то применяем первые 2 правила и последнее (по сути комбинируем случай 1 и 2)
		\item Если же они оба нечетны, то еще пользуемся $S \to a S b$
	\end{enumerate}

	Итого это нужная грамматика.

    \end{enumerate}


\end{document}
